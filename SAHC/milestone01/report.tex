\documentclass[a4paper, 11pt]{article}

\usepackage[utf8]{inputenc}
\usepackage[english]{babel}
\usepackage{a4wide}
\usepackage{indentfirst}
\usepackage{setspace}
\usepackage{relsize}
\usepackage{hyperref}
\usepackage[capitalise,noabbrev]{cleveref}
\usepackage{url}
\usepackage[normalem]{ulem}

\singlespace

\title{Security and Applications in Trusted Hardware \\ [0.8em] \smaller Portable Password Manager}
\author{João Silva \and Rui Fernandes \and Tiago Garcia}
\date{}

\renewcommand\labelitemi{--}

\hypersetup{
    pdftitle={Portable Password Manager},
    pdfborder={0 0 0}
}

\begin{document}

\maketitle

\section{Problem}

Current authentication schemes are based on something the user has (\textit{e.g.} a smart card or a 
security token), something the user knows (\textit{e.g.} a password or a PIN), or their physical 
characteristics (\textit{e.g.} a fingerprint or iris pattern) \cite{1246384}, being password-based 
authentication the most widely used.


One problem with password-based authentication is that infrequently used passwords are easy to 
forget, and it is hard to remember secure passwords. In fact, to avoid memorizing numerous 
passwords, users tend to re-use the same password on different platforms \cite{florencio2006a}. As 
a result, discovering a user's password grants an attacker access to multiple of the user's 
systems. Some solutions address this issue by requiring users to remember only one password. In 
\cref{sec:solutions}, we discuss two possible solutions to this problem, Single Sign On (SSO) and 
Password Managers.

\section{Existing Solutions} \label{sec:solutions}

In this section, we describe two different solutions to the problem of password re-use across 
multiple services, namely Single Sign On and Password Managers.

\subsection{Single Sign On}

User authentication can be delegated to an external service, known as Single Sign On (SSO). This 
service's purpose is to authenticate users and securely communicate successful authentication to 
the services the user wants to use. This way, the user only needs to remember the password he uses 
for the SSO service. Examples of SSO services include Shibboleth 
\cite{10.1007/978-3-642-37282-7_14}.

The main issue with this approach is their \uline{low portability} and their \uline{availability}, 
or lack thereof. For one thing, the service to which the user wants to authenticate must trust the 
SSO system. If this is not the case, the user still needs to remember a new password for the 
service. In other words, SSO systems only work with services that trust them.

On the other hand, a service may not trust in an SSO system due to the lack of availability 
guarantees. In other words, in case of a denial of service (DoS), the SSO may be disabled or 
rendered inaccessible, restricting user's ability to authenticate themselves.

\subsection{Password Managers}

Password managers are designed to store and provide access to a user's passwords. The user only 
needs to memorize a \textit{master password} and store his passwords in the password manager. 
Examples of password managers include Bitwarden \cite{Bitwarde48:online} and KeePass 
\cite{KeePassP1:online}.

End-to-end encryption ensures that only the user has access to their master password and password 
vault -- it cannot be accessed by the service provider or intercepted by hackers. However, since no 
one else can access the user's data, it is crucial that they remember the master password. 
Moreover, in order to safely store the passwords, symmetrical ciphers (\textit{e.g.} AES) are used. 
These algorithms use either the master password, or a transformation of it such as its hash value, 
as a key to encrypt the stored data. However, other solutions exist. For example, by default, 
Firefox does not require a master password, using a default key to cipher passwords data base.

Finally, despite being more resistant to DoS attacks, password managers are still susceptible to 
brute-force attacks.

\section{Proposed Solution}

The major problems with the existing solutions discussed in \cref{sec:solutions} are their 
portability and availability issues, and the fact that they are not resistant to brute-force 
attacks, or both. In our proposed solution, this is circumvented by using the smart card as a means 
of securely storing a cryptographic key which is used to cipher externally stored passwords. As 
such, the card is responsible for generating a symmetric cryptographic key (\textit{e.g.} AES) used 
to encrypt database containing the passwords, which will be stored in the cloud. Once the database 
is encrypted, we can assume that the passwords are stored securely. Whenever we want to access the 
password manager, the card reader will be responsible for storing and creating the passwords.

In this approach, the database could still be attacked, but the computational resources required to 
perform a successful attack are significantly greater given the size of the key. When it comes to 
access control, access to the card is protected by a PIN defined by the user, which cannot be 
brute-forced as the card will be locked after a certain number of unsuccessful attempts. In the 
case that the smart card supports multiple applications, no other application internal to the card 
can access the cryptographic key.

Finally, another factor that should be taken into account is that given the fact that a Java Card 
only persists data for about 10 years, a mechanism to export data to another card should be 
implemented.

\subsection{Trust Model}

In this work, we make the following assumptions:

\begin{itemize}
    \item The card is tamper-proof. Thus, it is not possible for an attacker to access the 
    cryptographic key without previous authentication.
    \item Only the user knows the smart card access PIN. This means that if somebody authenticates 
    with the user PIN, he is the expected user.
\end{itemize}

\pagebreak

\bibliographystyle{IEEEtran}
\bibliography{refs}

\end{document}
